\chapter{需求分析}
\label{chap:requirements-analysis}
在上一章绪论中,我们已经提到目前有大量的病历数据并没有电子化,或无法与现有的电子病历系统兼容协同,我们称之为未归档数据。从形成原因来说,未归档数据大致可以分为两大类:
\begin{itemize}
	\item 病历电子化之前的纸质档案,即现在很多医院的病历系统已经电子化,但是电子化之前的病历档案数据,仍然是纸质化的,这部分档案数据量庞大,包含了病人的历史诊疗记录,在临床上有很高的参考利用价值,故不能轻易舍弃,另一方面,旧的病历数据又很难与已有的电子化病历数据很好的协同使用,两者在档案产生、存储管理、信息展示、诊疗协助等各方面都有明显不同,为医院的诊疗工作进行提供了一些客观上的不便。
	\item 旧电子化系统中的档案,即一些医院存在着将旧的电子化病历系统废弃,采用新的电子化病历系统的情况,旧系统虽然废弃了,但是其保存的病人诊疗数据却是要保留和利用的。旧电子病历系统数据与纸质数据类似,也存在着与新电子化病历系统的协同问题,因此,如何将数据从旧系统向新系统迁移也是一个摆在面前的现实问题。
\end{itemize}
我们可以设想两个模拟的场景:
\begin{itemize}
	\item[场景A] 当一个病人去医院看病时,医生一边询问病人的主诉和症状,一边让助手去档案库中去找这个病人的过往诊疗记录,一边还要看医院电子档案系统的有关该病人的记录,纸质档案的寻找和放回都耗时耗力,面对
	\item[场景B] 
\end{itemize}
对比而言,场景A无疑是效率低下的,因此,如何将纸质病历数据和旧电子病历系统的数据迁移到新的电子化的病历系统中,是存在着广泛需求的,如果能够较好的解决这个问题,将能有效的改善现有医院病历系统的混乱现状,补全病人的医疗信息,提高诊疗效率,能给医院带来实际的效益。

\section{应用场景}
上一段中我们从总体上分析了病历档案生成与分类在医学领域的迫切需求和研究意义,现在我们更加细致的考虑这个问题,从系统角度来看,需求分析可以分别从系统环境、输入数据、输出数据以及性能这四个方面来论述。

\subsection{系统环境}
总体上,我们需要设计和实现一个病历档案生成和分类系统,系统在什么环境下运行呢?从成本、性能和稳定性等方面来综合考虑,这个系统的运行环境应该满足以下要求:
\begin{itemize}
	\item 系统能运行在普通的个人电脑上。这主要是出于成本的考虑,首先,个人电脑现在相当普及,易于获得,但是普遍性能较弱,稳定性没有保障。所以,系统应该能正常运行在性能较弱的个人电脑上。
	\item 系统能运行在商用服务器上。这主要是出于性能和稳定性上的考虑,服务器相对于个人电脑,性能更加强劲,同时,服务器有着高度的稳定性,整个系统可以很好的在服务器上长期稳定运行,持续产出,同时,服务器的一些特性也比较方便运维人员的维护和管理。
	\item 系统能对多种平台进行支持。医院内部采用的电子化系统可能运行在不同的平台上,对于主流平台,即Windows和Linux,以及Windows Server和Linux Server,系统都应该能够支持。同时,对于这些平台的不同版本(例如Windows 7和Windows XP),也要有较好的支持。
\end{itemize}

\subsection{输入数据}
\subsubsection{数据转换}
在这一章的开头,我们提到,待归档的数据主要有两类,纸质病历数据和旧电子化档案。在实际应用中,这两种数据都是不能直接被计算机系统识别的,因此,我们分别对其进行一定的转化:
\begin{itemize}
	\item 对于纸质数据,通过一台扫描仪或者照相机对病历进行录入,转换成图片数据。
	\item 对于旧电子化档案数据,我们可以在个人电脑上通过批处理的方式,打开旧电子病历,并利用个人电脑的截图功能,将病历转换成图片数据。
\end{itemize}
经过这样的转换后,这两种数据都已经转换成了图片数据,