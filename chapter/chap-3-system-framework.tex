\chapter{架构设计}
\label{chap:system-framework}
当系统需求和输入输出明确以后,首先要做的就是系统的架构设计,一个好的架构设计能让系统层次简明,易于理解,各构件各司其职,从而有效提高系统的效率、稳定性、安全性、扩展性和可维护性。

\section{框架设计}
从之前的需求分析来看,医学档案自动生成与归档系统从输入图片数据,到最终输出的文档数据,需经过若干个数据处理加工步骤,这些步骤之间是顺序依次执行的,并且模块之间没有向前的反馈关系,即模块间以线性展开,没有环路,这天然构成了系统架构设计中的管道模式(Pipeline Pattern)\citep{Vermeulen1995pipeline},或称过滤器模式(Filter Pattern)。管道/过滤器模式的软件架构,是系统架构中的一种非常常见的形式,每个构件都有一组输入输出,构件读输入的数据流,经过内部处理,然后产生输出数据流。这个过程通常通过对输入流的变换及增量计算来完成,所以在输入被完全消费之前,输出便产生了。管道/过滤器系统架构具有以下优点:
\begin{itemize}
	\item 允许设计者将整个系统的输入/输出行为看成是多个过滤器的行为的简单合成;
	\item 系统维护和增强系统性能简单。新的过滤器可以添加到现有系统中来;旧的可以被改进的过滤器替换掉;
	\item 允许对一些如吞吐量、死锁等属性的分析;
	\item 支持并行执行。每个过滤器是作为一个单独的任务完成,因此可与其它任务并行执行。
\end{itemize}
%这几句摘自http://blog.csdn.net/lizhengnanhua/article/details/14640489
具体到本系统来说,基于管道式的架构设计思想,我们可以将整个系统分成若干个处理模块,一组数据流将这些处理模块串联起来,数据流以每个病人的病历数据为单位,从输入口经过一系列加工处理,得到最终的输出,各个病人病历之间互不干扰,犹如生产线一般严格产出。

确定了采用管道式的系统架构后,接下来我们可以结合问题需求,按照数据流的顺序,一步步设计管道中的构件(模块),首先,我们需将输入的图片数据作为数据流的开端,让我们乘着数据的一叶扁舟,顺流而下,逐一分析与设计。

\section{模块设计}
\subsection{数据加载模块}
数据加载模块是数据流的入口,病人病历数据以多张图片的形式输入,对于计算机而言,是无法判断某张图片是属于哪个病人的,如果不对这些图片数据进行标识,很容易出现张冠李戴的现象,现实中如果将病人A的诊疗数据误录入到病人B的病历档案中,一方面病人A的诊疗数据就此丢失,医生做诊断的时候就没法做到全面,另一方面,更严重的是医生对病人B的诊断依据的是完全错的历史诊疗记录,很可能对病人B做出不恰当的诊断,这种误诊造成的伤害是不可估量的。因此,系统对输入数据应提出这样的需求,即每个病人的病历数据应该具有唯一的标识,如图片名字拥有唯一的前缀等。而数据加载模块,就可以根据这样的标识,来给海量的图片数据做好分门别类的工作,确保数据以病人为原子单位向后传递。

\subsection{版面分析模块}
当数据到达版面分析模块时,数据流中流淌的仍然是图片数据,这些图片数据中的每个“原子数据“是以病人为单位构成的一组图片集合,假设每个病人的病历数据对应着10张图片,这些图片间的病历信息互不重叠,是病历数据的有序记录,那么一个现实的问题是,当一组图片数据到来时,系统如何判断它们之间的次序,或者说某张图片中期望获得的字段数据有哪些呢?这就要求版面分析具有的第一个功能就是分辨这一组10张图片的总体属性,如果说数据加载模块是将图片数据以病人为单位做了第一步的划分,那么版面分析模块就是在此之上以图片为单位做的第二层划分。

版面分析模块的使命不止于此,当划分好每张图片后,图片中仍然存有着大量的冗余干扰信息,如\autoref{pic:input-image-1}中所示,图中除了需要识别的主题内容外,还存在着系统信息、侧边栏列表等无关信息,版面分析模块需要剔除这些干扰信息,分离出病历主题内容。

最后,对于提取出的病历主体内容,其实仍然存在着大片的空白,如果直接将它传递给下游的文字识别模块,文字相对于图片过于稀疏,识别准确率肯定大打折扣,而且识别出来的文字包含有多个字段,要从这些文字中正确解析各个字段,难度也不小。因此,版面分析模块还需要从主体内容中剔除空白内容,将各个字段提取出来,形成若干个子图片,这些子图片只包含文字区域,一个子图片对应一个字段,这样,技能有效提高后续文字识别模块的准确率,又能降低字段解析的难度,可谓一举两得。

从上面的分析来看,版面分析模块对于整个系统的性能表现起着至关重要的作用,它应该具备以下三个方面的功能:
\begin{itemize}
	\item 对病历图片进行二次分类;
	\item 剔除图片中的干扰信息,获得病历数据主体;
	\item 剔除主体内容中的空白,提取出各个字段。
\end{itemize}
不难发现,版面分析模块本身,也可以认为是一个管道模式架构设计。

\subsection{预处理模块}
提取出来的字段图片,虽然以文字为主体,但是文字本身很可能是比较小的,从\autoref{pic:input-image-1}中就不难看出,这样的文字大小对人来说尚且有一定的辨识难度,遑论机器了,如果直接将这些图片数据丢给文字识别系统进行识别,准确性可想而知,因此,我们需要对这些文字进行必要的预处理工作,在不改变文字本身线条和结构特性的前提下,尽量放大文字尺寸和它的轮廓特征,以便文字识别模块的识别,在光学字符识别领域,预处理的好坏往往能大大影响识别精度,好的预处理能让文字识别事半功倍,反之则会成为识别精度提升的瓶颈。

\subsection{OCR模块}
接下来自然来到了整个文字识别的核心:光学字符识别(OCR)模块。光学字符识别在印刷体识别领域的应用由来已久\citep{impedovo1991optical},技术也早已成熟,其本质是一个分类器,只是输入数据被限制为图片,输出数据为文本。当识别场景简单明确时,现有的OCR技术的识别精度已经高,但是当识别场景复杂,杂讯干扰较多时,识别效果会大打折扣。所以,如果要优化OCR模块的识别效果,一方面可以从优化分类器入手,如有针对性的设计数据集、增大训练样本等,另一方面,就是前置与处理模块,将识别问题简化,有关OCR的优化,在第四章中会有更加详细的论述。
数据流经过这一个模块之后,从图片转换成了文本。

\subsection{字段解析模块}
经过OCR模块的处理,图片转换成了文字,是不是意味着大功告成了呢?显然不是。首先一个问题,就是系统其实并不能从语义上理解识别出来的文字的,因此对于机器来说,这无非是从一种无法理解的数据转换成了另外一种无法理解的数据;其次,经过OCR模块识别出来的文本,带有某些冗余信息或者是错误信息,需要进行进一步的文本清洗和矫正工作。因此,我们需要加入一个字段解析模块,既负责文本的清洗矫正工作,又承担字段解析的功能。这个模块也可以成为“后处理模块”,是对识别文本的进一步优化。

\subsection{数据存储模块}
字段被解析出来了,那么应该如何存储呢?这就需要我们整个系统最后一个模块的引入:数据存储模块。在\autoref{chap:requirements-analysis}中我们提到,需求方期望的输出是有一定的数据组织形式的、有一定可读性的最终数据,数据存储模块就是承担着数据格式的组织和存储工作的,经过它的转化之后,得到最终的需求输出。

\section{小结}
至此,我们根据病历档案自动生成与分类问题的实际需求,完成了一个管道模式/过滤器模式(Pipeline Pattern/ Filter Pattern)的系统框架设计,系统整体框架如\autoref{pic:system-framework}
所示,包含六个模块,模块间由数据流完成连接。这个框架即有一定的理论支撑,又结合病历档案系统的实际,具备比较高的可行性。
\begin{figure}
	\centering
	\includegraphics[width=0.9\textwidth]{system-framework}
	\caption{系统的整体架构设计}
	\label{pic:system-framework}
\end{figure}
