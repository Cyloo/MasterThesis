\chapter{架构设计}
\label{chap:system-framework}
当系统需求和输入输出明确以后,首先要做的就是系统的架构设计,一个好的架构设计能让系统层次简明,易于理解,各构件各司其职,从而有效提高系统的效率、稳定性、安全性、扩展性和可维护性。

\section{主体设计}
从之前的需求分析来看,医学档案自动生成与归档系统从输入图片数据,到最终输出的文档数据,需经过若干个数据处理加工步骤,这些步骤之间是顺序依次执行的,并且模块之间没有向前的反馈关系,即模块间以线性展开,没有环路,这天然构成了系统架构设计中的管道模式(Pipeline Pattern),或称过滤器模式(Filter Pattern)。管道/过滤器模式的软件架构,是系统架构中的一种非常常见的形式,每个构件都有一组输入输出,构件读输入的数据流,经过内部处理,然后产生输出数据流。这个过程通常通过对输入流的变换及增量计算来完成,所以在输入被完全消费之前,输出便产生了。管道/过滤器系统架构具有以下优点:
\begin{itemize}
	\item 允许设计者将整个系统的输入/输出行为看成是多个过滤器的行为的简单合成;
	\item 系统维护和增强系统性能简单。新的过滤器可以添加到现有系统中来;旧的可以被改进的过滤器替换掉;
	\item 允许对一些如吞吐量、死锁等属性的分析;
	\item 支持并行执行。每个过滤器是作为一个单独的任务完成,因此可与其它任务并行执行。
\end{itemize}
%这几句摘自http://blog.csdn.net/lizhengnanhua/article/details/14640489
具体到本系统来说,基于管道式的架构设计思想,我们可以将整个系统分成若干个处理模块,一组数据流将这些处理模块串联起来,数据流以每个病人的病历数据为单位,从输入口经过一系列加工处理,得到最终的输出,各个病人病历之间互不干扰,犹如生产线一般严格产出。
确定了采用管道式的系统架构后,接下来我们可以结合问题需求,按照数据流的顺序,一步步设计管道中的构件(模块),首先,我们需将输入的图片数据作为数据流的开端,让我们乘着数据的一叶扁舟,顺流而下,逐一分析与设计。

\section{数据加载模块}

