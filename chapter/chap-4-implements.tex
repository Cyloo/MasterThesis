\chapter{实现细节}
\label{chap:implements}
前一章中,我们已经介绍了整个系统的架构设计,这一章我们会对系统各个模块的一些实现细节进行详细阐述,主要涉及到方案选取、环境部署、算法设计与实现、系统性能调优等。一方面是对整体架构设计的补充论述,另一方面,我们会更多的讨论在实现过程中遇到的各种问题,以及如何解决这些问题。整体上,我们会倾向去寻找已有的成熟解决方案,辅以适当的修改和优化。遵循这个原则,既保证了整个系统有一定的成熟度,又对病历数据的识别有较强的针对性。这一部分的阐述比较详细,按照这一章的指引,有一定计算机编程基础的读者,应该能实现整个系统的基本复现。

\section{数据加载模块}  %500字

\section{版面分析模块}  % 3000字
\subsection{方案选取}


\section{预处理模块}     % 1000字
\subsection{方案选取}


\section{OCR模块}     % 6000字
\subsection{方案选取}

\section{字段解析模块}  %3000字
当含有文本的图片数据经过OCR模块以后,会被转换为文本,这些文本需要进行一定的加工,才能得到我们想要的各字段数据。总体来说,原始的文本需要经过数据清洗、字段匹配、目标文本提取等步骤。我们也可以认为这是数据的“后处理”过程。

\subsection{数据清洗}
数据清洗主要有两大功能,一是对文本的冗余信息(如因为图片噪声而产生的冗余符号)进行剔除,二是对文本进行简单的矫正。这两部分实现难度不大,却对之后的字段匹配过程有着很大的帮助。
\subsubsection{冗余信息剔除}
\subsubsection{文本矫正}

\subsection{字段匹配}
某段文本属于哪一个字段呢?这就需要对文本的具体内容进行理解划分了,例如若要判断
\subsubsection{StartsWith匹配}
\subsubsection{Includes匹配}
	

\section{数据存储模块} %500字
