\chapter{实现细节}
这一章介绍整个系统的各个模块的一些实现细节。主要涉及到方案选取、环境部署、算法设计与实现等。
\section{版面分析模块}
\section{预处理模块}
\subsection{方案选取}


\section{OCR模块}
\subsection{方案选取}

\section{字段解析模块}
当含有文本的图片数据经过OCR模块以后,会被转换为文本,这些文本需要进行一定的加工,才能得到我们想要的各字段数据。总体来说,原始的文本需要经过数据清洗、字段匹配、目标文本提取等步骤。我们也可以认为这是数据的“后处理”过程。

\subsection{数据清洗}
数据清洗主要有两大功能,一是对文本的冗余信息(如因为图片噪声而产生的冗余符号)进行剔除,二是对文本进行简单的矫正。这两部分看上去简单,却对之后的字段匹配过程有着很大的帮助。
\subsubsection{冗余信息剔除}
\subsubsection{文本矫正}


\subsection{字段匹配}
某段文本属于哪一个字段呢?这就需要对文本的具体内容进行理解划分了,例如若要判断
\subsubsection{StartsWith匹配}
\subsubsection{Includes匹配}


\section{数据存储模块}
