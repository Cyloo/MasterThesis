\chapter{实验}
\label{chap:experiments}
本章将对病历档案自动归档与分类系统进行完整的系统测试,验证其是否满足了\autoref{chap:requirements-analysis}中提出的各种需求。首先会介绍实验的运行环境,包括软件和硬件环境。然后,测试系统的功能是否完整及其性能表现,以及通过对比实验来测试各项优化带来的性能提升。
\section{实验环境}
本系统支持在多平台下部署,作为测试环境,选用的是Windows平台(Windows 7,X64),使用Visual Studio 2013进行开发与测试。本系统通过读取本地硬盘中特定方式组织与存储的病历图片数据,采用图像处理和OCR技术,转换为特定组织形式的文本数据,供进一步电子化和数据入库操作。系统的软硬件测试环境如\autoref{experiment-environment} 所示。
\begin{table}[!htbp]
  \tabcaption{系统实验环境}
	\label{experiment-environment}
	\centering
	\vspace{10pt}
  \renewcommand\arraystretch{1.5}  %这两行代码用来调整表格高度
	\begin{tabularx}{\textwidth}{C|C|C|C}
	\hline
    \multicolumn{2}{c|}{硬件环境} & \multicolumn{2}{c}{软件环境}\\

	\hline
	CPU&GPU&操作系统&工具库 \\

    \hline
    %\cmidrule(r){1} \cmidrule(r){2} \cmidrule(r){3} \cmidrule(r){4}
	\begin{tabular}[]{c}
		%Intel(R) Core(TM) \\
		Intel \textsuperscript{\textregistered} Core\texttrademark \\
		i7-4770K @ \\
		3.50GHz \\
	\end{tabular}
	&
	\begin{tabular}[]{c}
		NVIDIA  \\
	 	GeForce GTX  \\
	 	760 \\
	\end{tabular}
	&
	\begin{tabular}[]{c}
		Microsoft  \\
	 	Windows 7 \\
		x64 \\
	\end{tabular}
	&
	\begin{tabular}[]{c}
	OpenCV \\
	Tesseract\\
	\end{tabular}\\

	\hline


	%\end{tabular}
	\end{tabularx}
\end{table}


\section{实验设计}
\label{sec:experiment-design}
之前的分析已经阐明,病历档案自动归档与分类系统中,文字识别只是其中的一个模块,另外还有版面分析模块,字段提取模块等,所以在测试系统性能时,需要从版面分析、字段提取、文字识别三个方面来考量。对于一个病例档案归档系统来说,最核心的性能指标无疑是识别准确率,前文提到,系统针对这三个方面都有针对性优化来提升准确率,因此,在接下来的实验中,我们将逐步验证各项优化带来的准确率差异。另外,也会从整体上测试系统的性能表现,根据这些实验数据来综合判断系统是否已经达到设计阶段所提出的各项功能和性能需求。
\subsection*{测试数据}
本次实验的测试数据是随机选取的8份病历数据,每份病历数据包含7到8张图片,这7到8张图片可以分为三种版面,每份病历数据中又包含约60个字段,包含有中英文、数字等字符2500个左右。因此总的训练数据共有62张图片,三种主要版面,500个左右的字段,20000个左右的字符需要识别。这种量级的测试数据具有一定的代表性和说服力。
\section{优化效果对比}
这一章节主要考察各项优化对于各模块的优化效果,这些优化效果主要从准确率方面进行比较,而有关各个模块的耗时情况,在整体性能测试中会涉及到。
\subsection{版面分析优化}
由于大部分OCR引擎普遍在版面分析方面比较薄弱,尤其是对于特定场景下的版面分析更是乏力,因此,在选取版面分析的基础对比组的时候,我们选用的是目标图片与模板图片逐个像素对比的方法作为实验的“未优化方法”,这种方法首先效率就极为低下,实验中暂且忽略效率上的差异,仅从准确率方面考量,观察本系统的版面分析模块所带来的效果差异。
\begin{table}[!htbp]
  \tabcaption{版面分析优化效果对比}
	\label{layout-analysis-optimization}
	\centering
	\vspace{10pt}
  \renewcommand\arraystretch{1.5}  %这两行代码用来调整表格高度
	\begin{tabular}{c||c|c|c|c}
    \hline
    & 正确数 & 错误数 & 总数 & 准确率 \\
		\hline
    未优化方法 & 44 & 18 & 62 & 71.0\% \\
		\hline
    优化后方法 & 62 & 0 & 62 & 100\% \\
    \hline
	\end{tabular}
\end{table}

实验结果如\autoref{layout-analysis-optimization}所示,从对比实验中可以看到,经过版面分析模块优化之后,系统对版面的分析能力大大加强,优化后的版面匹配准确率达到了100\%,高准确率一方面有本系统的版面种类较少等客观原因,另外,也主要得益于以下几个优化:
\begin{itemize}
  \item 通过模板匹配的方法,截去了图片中的系统信息栏和侧边栏等干扰区域,提取了病历数据的主体;
  \item 通过灰度化,二值化,腐蚀等方法,抹去了目标图片与模板图片在文字细节上的差异,突出了版面特征;
  \item 通过将二值图片的黑白像素反转,并用图片矩阵点乘的方式作为相似性度量,高效而准确的判断版面相似性。
\end{itemize}

\subsection{字段提取优化}
在\autoref{sec:experiment-design}中提到测试数据中对于每份病历包含约60个字段需要识别,这些字段分布在图片的各个位置,本系统通过图像的二值化、腐蚀等操作,将各个字段变成独立的连通域,之后再经过连通域分析,将各字段分割提取出来进行文字识别。本实验中,我们将整个图片直接识别成大段文字,然后从文本中提取相应字段,作为实验的对比方法(“未优化方法”),观察本系统的字段提取模块所带来的效果差异。
\begin{table}[!htbp]
  \tabcaption{字段提取优化效果对比}
	\label{tokens-extraction}
	\centering
	\vspace{10pt}
  \renewcommand\arraystretch{1.5}  %这两行代码用来调整表格高度
	\begin{tabular}{c||c|c|c|c}
    \hline
    & 正确数 & 错误数 & 总数 & 准确率 \\
		\hline
    未优化方法 & 325 & 147 & 472 & 68.9\% \\
		\hline
    优化后方法 & 443 & 29 & 472 & 93.9\% \\
    \hline
	\end{tabular}
\end{table}

实验结果如\autoref{tokens-extraction}所示,可以看到,本系统在字段提取方面的准确率上明显优于未优化方法。准确率提升主要得益于以下几项优化:
\begin{itemize}
  \item 图像的二值化、腐蚀、连通域分析方法将各个字段很好的分割开来,使得各个字段的边界非常明确,便于提取;
  \item 每个字段分割开来之后,文字成为了各个字段图片的主体,使得识别准确率得到提升。
\end{itemize}

\subsection{文字识别优化}
对于具体的文本识别,系统从预处理、训练字符库生成、用户字典以及后处理等多方面进行了识别准确性优化,这部分实验设计中,我们分别去除各个单项优化,构成实验组,与综合优化后的结果(作为对照组)做准确性的比较,来观察各单项优化的效果。
\begin{table}[!htbp]
  \tabcaption{文字识别优化效果对比}
	\label{ocr-optimization}
	\centering
	\vspace{10pt}
  \renewcommand\arraystretch{1.5}  %这两行代码用来调整表格高度
	\begin{tabular}{c||c|c|c|c}
    \hline
    & 正确数 & 错误数 & 总数 & 准确率 \\
		\hline
    综合优化方法 & 18420 & 1957 & 20377 & 90.4\% \\
		\hline
    去除用户字典 & 17625 & 2752 & 20377 & 86.5\% \\
		\hline
    去除后处理 & 17937 & 2440 & 20377 & 87.9\% \\
		\hline
    去除医疗训练集 & 16812 & 3565 & 20377 & 82.5\% \\
		\hline
    去除图像预处理 & 13082 & 7295 & 20377 & 64.2\% \\
		\hline
    去除所有优化 & 8652 & 11725 & 20377 & 42.5\% \\
    \hline
	\end{tabular}
\end{table}
实验结果如\autoref{ocr-optimization}所示,从实验结果中我们可以看到,各项优化均对最终的准确率有不同程度的提升作用,其中,图片的预处理对于准确率的提升有着最为明显的作用,自训练字符库次之,由此可知,在文字识别系统中,图片的预处理对于最终的识别准确率有着至关重要的影响。
另外,当去除所有优化以后,准确率仅为42.5\%,由此可见各项优化所带来的巨大准确率提升。

\section{整体性能测试}
整体性能测试是在综合系统各项优化以后进行的测试,它包括了版面匹配、字段提取、文字识别各部分的准确率和耗时情况,能比较直观的反映系统的整体性能,具体结果可见\autoref{overall-performance}。
从实验结果可知,系统在各个子模块的准确率都达到了较高的水平,由于现在商用的OCR软件普遍没有针对性比较高的版面分析和字段提取解决方案,无法与本系统的对应模块进行比较,而仅从文字识别的精度来看,本系统已经与成熟的商用OCR软件的识别精度相近甚至更高。
而从耗时情况来看,系统处理一份病历数据,平均需耗时3分钟左右,其中大部分的耗时都发生在文字识别阶段,这个阶段内系统除了需要进行OCR文字识别外,还需要对文字进行预处理、后处理等工作,确实会消耗比较多的时间。总体而言,本系统基本满足了设计时的各项功能和性能需求。
\begin{table}[!htbp]
  \tabcaption{综合性能测试}
	\label{overall-performance}
	\centering
	\vspace{10pt}
  \renewcommand\arraystretch{1.5}  %这两行代码用来调整表格高度
	\begin{tabular}{c|c|c|c|c|c|c}
    \hline
    模块 & 正确 & 错误 & 总数 & 正确率 & 总耗时(s) & 平均耗时(s/份) \\
    \hline
    版面分析 & 62 & 0 & 62 & 100\% & 20.3 & 2.5 \\
    \hline
    字段提取 & 443 & 29 & 472 & 93.9\% & 31.2 & 3.9 \\
    \hline
    文字识别 & 18420 & 1957 & 20377 & 90.4\% & 1634.4 & 204.3 \\
    \hline
	\end{tabular}
\end{table}

\section{小结}
本章对病历档案自动归档与分类系统进行了详细的功能和性能测试。通过对比实验,展现了系统的各个模块的优化效果,并简要分析了其原因。另外,通过整体性能测试,证明了本系统基本满足了设计阶段的各项功能和性能需求,系统已经具备了较高的识别精度、可接受的运行效率,有高度的实用性和可靠性。
