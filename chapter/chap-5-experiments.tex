\chapter{实验}
\label{chap:experiments}
本章将对病历档案自动归档与分类系统进行完整的系统测试,验证其是否满足了\autoref{chap-2-requirement-analysis}中提出的各种需求。首先会介绍实验的运行环境,包括软件和硬件环境。然后,测试系统的功能是否完整及其性能表现,以及通过对比实验来测试各项优化带来的性能提升。
\section{实验环境}
本系统支持在多平台下部署,作为测试环境,选用的是Windows平台(Windows 7,X64),使用Visual Studio 2013进行开发与测试。本系统通过读取本地硬盘中特定方式组织与存储的病历图片数据,采用图像处理和OCR技术,转换为特定组织形式的文本数据,供进一步电子化和数据入库操作。系统的软硬件测试环境如\autoref{experiment-environment} 所示。
\begin{table}[!htbp]
  \tabcaption{系统实验环境}
	\label{experiment-environment}
	\centering
	\vspace{10pt}
  \renewcommand\arraystretch{1.5}  %这两行代码用来调整表格高度
	\begin{tabular}{p{3cm}|p{3cm}|p{3cm}|p{3cm}}
    \hline
    \multicolumn{2}{c|}{硬件环境} & \multicolumn{2}{c}{软件环境} \\
		\hline
    CPU&GPU&操作系统&工具库 \\
		\hline
    Intel(R) Core(TM)  i7-4770K @ 3.50GHz & NVIDIA GeForce  GTX 760 & Microsoft  Windows 7 x64 & OpenCV  Tesseract \\
    \hline
	\end{tabular}
\end{table}

\section{实验设计}
之前的分析已经阐明,病历档案自动归档与分类系统中,文字识别只是其中的一个模块,另外还有版面分析模块,字段提取模块等,所以在测试系统性能时,需要从版面分析、字段提取、文字识别三个方面来考量。前文提到,系统针对这三个方面都有针对性优化来提升准确率,因此,在接下来的实验中,我们将逐步验证各项优化带来的准确率差异。另外,也会从整体上测试系统的性能表现。
\subsection*{测试数据}
本次实验的测试数据是随机选取的8份病历数据,每份病历数据包含7到8张图片,这7到8张图片可以分为三种版面,每份病历数据中又包含约60个字段,包含有中英文、数字等字符2500个左右。因此总的训练数据共有62张图片,三种主要版面,500个左右的字段,20000个左右的字符需要识别。这种量级的测试数据具有一定的代表性和说服力。
\section{优化效果对比}
这一章节主要考察各项优化对于各模块的优化效果。
\subsection{版面分析模块}
由于大部分OCR引擎普遍在版面分析方面比较薄弱,尤其是对于特定场景下的版面分析更是乏力,因此,在选取版面分析的基础对比组的时候,我们选用的是目标图片与模板图片逐个像素对比的方法作为实验的“未优化方法”,这种方法首先效率就极为低下,实验中暂且忽略效率上的差异,仅从准确率方面考量,观察本系统的版面分析模块的效果差异。实验结果如
\subsection{自训练字符库优化}

\subsection{用户字典优化}
\section{整体性能测试}

\section{小结}
