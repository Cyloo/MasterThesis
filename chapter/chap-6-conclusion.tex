\chapter{结论}
\label{chap:conclusion}
\section*{总结}
本文首先介绍了病历档案自动归档与分类系统的研究背景和意义,从实际应用场景出发,深入分析并明确了系统的各项功能和性能需求。
在此之上,本文设计了一个多模块、松耦合的流水线式的系统整体框架,使得系统具有高效、高扩展性的特点。
在实现细节上,系统基于开源图片处理库OpenCV和开源字符识别引擎Tesseract,加入版面分析、医疗数据训练集、数据预处理和后处理等的支持,实现了一个针对真实应用场景的印刷体病历档案自动归档与分类系统。
实验证明,系统在版面分析、字段提取、文字识别等方面都有着较高的准确率,具备较高的可靠性和实用价值。

本文核心工作可以归纳为:
\begin{itemize}
  \item 针对特定的病历图片数据,通过图像处理技术,设计和实现了完善的版面分析模块,填补了大部分商用OCR软件应用到病历数据时在版面分析能力的缺失。
  \item 基于开源OCR引擎Tesseract,针对病历数据,增加了图片预处理、医疗数据训练集、文本后处理等的支持,大大提高了文本识别的准确率。
\end{itemize}

\section*{后续工作}
对于医学档案自动归档与分类系统来说,最核心的性能指标就是识别准确率和处理速度。
在系统开发和测试过程中,本系统的文字识别准确率和处理速度虽然都达到了预期的设计要求,但是都有一定的提升空间。
在后续的工作中,将从这两方面来提升系统的性能。
\subsection*{准确率提升}
虽然Tesseract支持加入用户字典,来加强对高频词的识别准确率,但是从实验中可以看到,用户字典的加入对于性能的提升很有限,说明Tesseract在字典信息的利用上并不是很优秀,本系统可从这方面着手提高识别准确率。

事实上,Tesseract提供了识别候选字接口,即对于每个字,Tesseract在识别过程中会产生多个候选字,并给出每个候选字的置信度,再根据上下文信息,用户字典等,给出每个候选字的综合打分,保留综合得分最高的字作为最后的输出。
那么,一个可能的优化方案是系统拿到各个字的识别候选字及其置信度以后,根据统计语言模型(Statistical Language Modeling)\citep{brown1990statistical}建模,得到最可能的候选字组合,这种方法也许能有效提高文字识别准确率。
\subsection*{速度提升}
从实验中可以看到,本系统在处理速度上主要受限于Tesseract的文字识别速度,其他模块的优化提速并不会明显提高系统的整理处理速度。
一个简单可行的提升方法是将本系统部署到多机环境中,只要做好数据读取和存储时的协同,理论上处理速度会随着机器数据线性增长。
